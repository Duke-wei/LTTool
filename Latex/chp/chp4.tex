\chapter{数据结构}
\section{堆}
\subsection{堆排序事件空间复杂度}
堆排序包括初始化建立堆和后续排序的更新堆操作,初始化建立堆的平均时间复杂度为O(n),而因为更新结点需要维护堆的事件复杂读为$O(log^N)$,因此堆排序的事件复杂读为$O(nlog^N)$,空间复杂度为O(1)。
\subsection{堆建立}
一般通过数组来存取二插堆,其中i的父节点为$(i-1)/2$,i的左右子结点为$2*i+1$,$2*i+2$。初始化构建堆的方法是从中间结点开始向下调整。
\subsection{堆性质}
数据的逻辑结构分为线性结构和非线性结构。\\
常用的线性结构有:线性表、栈、队列、双队列、数组、串\\
常用的非线性结构有:二维数组、多维数组、广义表、树、图、堆
\subsection{堆排序的稳定性}
堆排序是不稳定的排序算法。\\
常见的不稳定排序算法有:堆排序、快速排序、希尔排序、简单选择排序\\
常见的稳定排序算法有:基数排序、冒泡排序、简单插入排序、归并排序
\section{字符串}
串是一种特殊的线性表,其特殊性体现在数组元素是一个字符。
